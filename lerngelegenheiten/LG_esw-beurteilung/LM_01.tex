% Options for packages loaded elsewhere
\PassOptionsToPackage{unicode}{hyperref}
\PassOptionsToPackage{hyphens}{url}
\PassOptionsToPackage{dvipsnames,svgnames,x11names}{xcolor}
%
\documentclass[
  10pt,
  a4paper,
  DIV=11]{scrartcl}

\usepackage{amsmath,amssymb}
\usepackage{iftex}
\ifPDFTeX
  \usepackage[T1]{fontenc}
  \usepackage[utf8]{inputenc}
  \usepackage{textcomp} % provide euro and other symbols
\else % if luatex or xetex
  \usepackage{unicode-math}
  \defaultfontfeatures{Scale=MatchLowercase}
  \defaultfontfeatures[\rmfamily]{Ligatures=TeX,Scale=1}
\fi
\usepackage{lmodern}
\ifPDFTeX\else  
    % xetex/luatex font selection
  \setmainfont[]{Arial}
\fi
% Use upquote if available, for straight quotes in verbatim environments
\IfFileExists{upquote.sty}{\usepackage{upquote}}{}
\IfFileExists{microtype.sty}{% use microtype if available
  \usepackage[]{microtype}
  \UseMicrotypeSet[protrusion]{basicmath} % disable protrusion for tt fonts
}{}
\makeatletter
\@ifundefined{KOMAClassName}{% if non-KOMA class
  \IfFileExists{parskip.sty}{%
    \usepackage{parskip}
  }{% else
    \setlength{\parindent}{0pt}
    \setlength{\parskip}{6pt plus 2pt minus 1pt}}
}{% if KOMA class
  \KOMAoptions{parskip=half}}
\makeatother
\usepackage{xcolor}
\usepackage[top=30mm,left=20mm,heightrounded]{geometry}
\setlength{\emergencystretch}{3em} % prevent overfull lines
\setcounter{secnumdepth}{3}
% Make \paragraph and \subparagraph free-standing
\ifx\paragraph\undefined\else
  \let\oldparagraph\paragraph
  \renewcommand{\paragraph}[1]{\oldparagraph{#1}\mbox{}}
\fi
\ifx\subparagraph\undefined\else
  \let\oldsubparagraph\subparagraph
  \renewcommand{\subparagraph}[1]{\oldsubparagraph{#1}\mbox{}}
\fi


\providecommand{\tightlist}{%
  \setlength{\itemsep}{0pt}\setlength{\parskip}{0pt}}\usepackage{longtable,booktabs,array}
\usepackage{calc} % for calculating minipage widths
% Correct order of tables after \paragraph or \subparagraph
\usepackage{etoolbox}
\makeatletter
\patchcmd\longtable{\par}{\if@noskipsec\mbox{}\fi\par}{}{}
\makeatother
% Allow footnotes in longtable head/foot
\IfFileExists{footnotehyper.sty}{\usepackage{footnotehyper}}{\usepackage{footnote}}
\makesavenoteenv{longtable}
\usepackage{graphicx}
\makeatletter
\def\maxwidth{\ifdim\Gin@nat@width>\linewidth\linewidth\else\Gin@nat@width\fi}
\def\maxheight{\ifdim\Gin@nat@height>\textheight\textheight\else\Gin@nat@height\fi}
\makeatother
% Scale images if necessary, so that they will not overflow the page
% margins by default, and it is still possible to overwrite the defaults
% using explicit options in \includegraphics[width, height, ...]{}
\setkeys{Gin}{width=\maxwidth,height=\maxheight,keepaspectratio}
% Set default figure placement to htbp
\makeatletter
\def\fps@figure{htbp}
\makeatother
% definitions for citeproc citations
\NewDocumentCommand\citeproctext{}{}
\NewDocumentCommand\citeproc{mm}{%
  \begingroup\def\citeproctext{#2}\cite{#1}\endgroup}
\makeatletter
 % allow citations to break across lines
 \let\@cite@ofmt\@firstofone
 % avoid brackets around text for \cite:
 \def\@biblabel#1{}
 \def\@cite#1#2{{#1\if@tempswa , #2\fi}}
\makeatother
\newlength{\cslhangindent}
\setlength{\cslhangindent}{1.5em}
\newlength{\csllabelwidth}
\setlength{\csllabelwidth}{3em}
\newenvironment{CSLReferences}[2] % #1 hanging-indent, #2 entry-spacing
 {\begin{list}{}{%
  \setlength{\itemindent}{0pt}
  \setlength{\leftmargin}{0pt}
  \setlength{\parsep}{0pt}
  % turn on hanging indent if param 1 is 1
  \ifodd #1
   \setlength{\leftmargin}{\cslhangindent}
   \setlength{\itemindent}{-1\cslhangindent}
  \fi
  % set entry spacing
  \setlength{\itemsep}{#2\baselineskip}}}
 {\end{list}}
\usepackage{calc}
\newcommand{\CSLBlock}[1]{\hfill\break\parbox[t]{\linewidth}{\strut\ignorespaces#1\strut}}
\newcommand{\CSLLeftMargin}[1]{\parbox[t]{\csllabelwidth}{\strut#1\strut}}
\newcommand{\CSLRightInline}[1]{\parbox[t]{\linewidth - \csllabelwidth}{\strut#1\strut}}
\newcommand{\CSLIndent}[1]{\hspace{\cslhangindent}#1}

%% load packages
\usepackage{sectsty}
\usepackage{fontspec}
\usepackage{scrlayer-scrpage}
\usepackage{fontspec}

% Define title formatting

%% Set font sizes for sections and subsections
\sectionfont{\fontsize{11}{16}\selectfont\bfseries} % Arial 14pt bold
\subsectionfont{\fontsize{10}{14}\selectfont} % Arial 12pt

\lehead[test1]{\pagemark}
\cehead[]{}
\rehead[test 2]{\leftmark}
\lohead[{\fontsize{8}{10}\selectfont\upshape\textbf{Institut Sekundarstufe I}} \\
\fontsize{8}{10}\selectfont\upshape Fabrikstrasse 8, CH-3012 Bern \\
\fontsize{8}{10}\selectfont\upshape T +41 31 309 21 15, contactdesk@phbern.ch, www.phbern.ch
]{\rightmark}
\rohead[PHBern]{}
\KOMAoption{captions}{tableheading}
\makeatletter
\@ifpackageloaded{tcolorbox}{}{\usepackage[skins,breakable]{tcolorbox}}
\@ifpackageloaded{fontawesome5}{}{\usepackage{fontawesome5}}
\definecolor{quarto-callout-color}{HTML}{909090}
\definecolor{quarto-callout-note-color}{HTML}{0758E5}
\definecolor{quarto-callout-important-color}{HTML}{CC1914}
\definecolor{quarto-callout-warning-color}{HTML}{EB9113}
\definecolor{quarto-callout-tip-color}{HTML}{00A047}
\definecolor{quarto-callout-caution-color}{HTML}{FC5300}
\definecolor{quarto-callout-color-frame}{HTML}{acacac}
\definecolor{quarto-callout-note-color-frame}{HTML}{4582ec}
\definecolor{quarto-callout-important-color-frame}{HTML}{d9534f}
\definecolor{quarto-callout-warning-color-frame}{HTML}{f0ad4e}
\definecolor{quarto-callout-tip-color-frame}{HTML}{02b875}
\definecolor{quarto-callout-caution-color-frame}{HTML}{fd7e14}
\makeatother
\makeatletter
\@ifpackageloaded{caption}{}{\usepackage{caption}}
\AtBeginDocument{%
\ifdefined\contentsname
  \renewcommand*\contentsname{Inhaltsverzeichnis}
\else
  \newcommand\contentsname{Inhaltsverzeichnis}
\fi
\ifdefined\listfigurename
  \renewcommand*\listfigurename{Abbildungsverzeichnis}
\else
  \newcommand\listfigurename{Abbildungsverzeichnis}
\fi
\ifdefined\listtablename
  \renewcommand*\listtablename{Tabellenverzeichnis}
\else
  \newcommand\listtablename{Tabellenverzeichnis}
\fi
\ifdefined\figurename
  \renewcommand*\figurename{Abbildung}
\else
  \newcommand\figurename{Abbildung}
\fi
\ifdefined\tablename
  \renewcommand*\tablename{Tabelle}
\else
  \newcommand\tablename{Tabelle}
\fi
}
\@ifpackageloaded{float}{}{\usepackage{float}}
\floatstyle{ruled}
\@ifundefined{c@chapter}{\newfloat{codelisting}{h}{lop}}{\newfloat{codelisting}{h}{lop}[chapter]}
\floatname{codelisting}{Listing}
\newcommand*\listoflistings{\listof{codelisting}{Listingverzeichnis}}
\makeatother
\makeatletter
\makeatother
\makeatletter
\@ifpackageloaded{caption}{}{\usepackage{caption}}
\@ifpackageloaded{subcaption}{}{\usepackage{subcaption}}
\makeatother
\ifLuaTeX
\usepackage[bidi=basic]{babel}
\else
\usepackage[bidi=default]{babel}
\fi
\babelprovide[main,import]{ngerman}
\ifPDFTeX
\else
\babelfont{rm}[]{Arial}
\fi
% get rid of language-specific shorthands (see #6817):
\let\LanguageShortHands\languageshorthands
\def\languageshorthands#1{}
\ifLuaTeX
  \usepackage{selnolig}  % disable illegal ligatures
\fi
\usepackage{bookmark}

\IfFileExists{xurl.sty}{\usepackage{xurl}}{} % add URL line breaks if available
\urlstyle{same} % disable monospaced font for URLs
\hypersetup{
  pdftitle={Thematische Einführung},
  pdfauthor={Richard Conrardy},
  pdflang={de},
  colorlinks=true,
  linkcolor={(3,1.5)},
  filecolor={Maroon},
  citecolor={Blue},
  urlcolor={Blue},
  pdfcreator={LaTeX via pandoc}}

\title{Thematische Einführung}
\usepackage{etoolbox}
\makeatletter
\providecommand{\subtitle}[1]{% add subtitle to \maketitle
  \apptocmd{\@title}{\par {\large #1 \par}}{}{}
}
\makeatother
\subtitle{Grundlagen der Beurteilung}
\author{Aline Löw \and Daniel Ingrisani \and Irene Althaus}
\date{}

\begin{document}
\maketitle
\begin{abstract}
In diesem Lernmodul erhalten Sie einen ersten Einblick in die Thematik
«Beurteilung und Förderung». Der Dokumentarfilm «Mein Leben und der
Notenschnitt» von Luzius Wespe zeigt auf eindrückliche Weise, was Kinder
beim Übertritt von der Primar- in die Sekundarstufe bewegt und was
«Noten» bei den Lernenden selber aber auch zu Hause in den Familien
auslösen können.
\end{abstract}

\begin{tcolorbox}[enhanced jigsaw, title=\textcolor{quarto-callout-note-color}{\faInfo}\hspace{0.5em}{Ziele des Lernmoduls}, colframe=quarto-callout-note-color-frame, colbacktitle=quarto-callout-note-color!10!white, left=2mm, leftrule=.75mm, arc=.35mm, toptitle=1mm, bottomrule=.15mm, breakable, opacityback=0, bottomtitle=1mm, coltitle=black, titlerule=0mm, opacitybacktitle=0.6, rightrule=.15mm, colback=white, toprule=.15mm]

Die Studierenden \ldots{}

\begin{itemize}
\tightlist
\item
  lernen Gefühle und Gedanken bezüglich des Notendrucks von Kindern
  kennen.
\item
  setzen sich mit der Sinnhaftigkeit und Notwendigkeit von Noten bzw.
  von Beurteilung auseinander.
\item
  entwickeln ein Problembewusstsein in Bezug auf Noten und
  Leistungsbewertung.
\end{itemize}

\end{tcolorbox}

\section{Mein Leben und die Noten}\label{mein-leben-und-die-noten}

\begin{tcolorbox}[enhanced jigsaw, title=\textcolor{quarto-callout-caution-color}{\faFire}\hspace{0.5em}{Aufträge auf LearningView}, colframe=quarto-callout-caution-color-frame, colbacktitle=quarto-callout-caution-color!10!white, left=2mm, leftrule=.75mm, arc=.35mm, toptitle=1mm, bottomrule=.15mm, breakable, opacityback=0, bottomtitle=1mm, coltitle=black, titlerule=0mm, opacitybacktitle=0.6, rightrule=.15mm, colback=white, toprule=.15mm]

\begin{figure}[H]

\centering{

\includegraphics[width=0.5\textwidth,height=\textheight]{images/kind_ki.jpg}

}

\caption{\label{fig-kind}Kind erhält Beurteilung ©Voltafilm}

\end{figure}%

Bevor Sie den Film schauen: Überlegen Sie sich kurz, was diesem Kind in
Abbildung~\ref{fig-kind} gerade durch den Kopf geht.

\end{tcolorbox}

Schauen Sie sich nun den Film
\href{https://phbern365.sharepoint.com/:v:/r/sites/IS1.Z.Studienplanentwicklung2022/Freigegebene\%20Dokumente/General/BA\%20-\%20Beurteilung\%20(formativ\%20und\%20summativ)/Lerngelegenheiten/Grundlagen\%20der\%20Beurteilung\%20SOL\%20-\%20Conrardy/SRF\%20DOK\%202021\%20-\%20Mein\%20Leben\%20und\%20der\%20Notenschnitt\%20\%E2\%80\%93\%20Vom\%20U\%CC\%88bertritt\%20in\%20die\%20Oberstufe.mp4?csf=1&web=1&e=4cFyOu}{Mein
Leben und der Notenschnitt} von Wespe (2022) an.

Wenn Sie sich von diesem Film haben zum Nachdenken anregen und aufwühlen
lassen, geht es in einem nächsten Schritt darum ein Problembewusstsein
für die Problematik der Notengebung und der Leistungsbewertung zu
entwickeln.

\begin{tcolorbox}[enhanced jigsaw, title=\textcolor{quarto-callout-caution-color}{\faFire}\hspace{0.5em}{Aufträge auf LearningView}, colframe=quarto-callout-caution-color-frame, colbacktitle=quarto-callout-caution-color!10!white, left=2mm, leftrule=.75mm, arc=.35mm, toptitle=1mm, bottomrule=.15mm, breakable, opacityback=0, bottomtitle=1mm, coltitle=black, titlerule=0mm, opacitybacktitle=0.6, rightrule=.15mm, colback=white, toprule=.15mm]

Machen Sie sich dabei Notizen, indem Sie unter anderem auch versuchen
Antworten auf die folgenden Fragen zu finden.

\begin{itemize}
\tightlist
\item
  Wie sind Sie selber mit Noten in Ihrer Schulzeit umgegangen?
\item
  Wie hat ihr Umfeld auf gute/schlechte Noten reagiert?
\item
  Was für eine Einstellung gegenüber Noten hatten Lehrpersonen aus Ihrer
  Vergangenheit?
\item
  Was für eine Haltung gegenüber Noten haben Sie als (zukünftige)
  Lehrperson?
\end{itemize}

\end{tcolorbox}

In den nun folgenden Lernmodulen, die Sie im Verlaufe des Semesters
durcharbeiten werden, soll es immer wieder auch darum gehen, Antworten
zu finden und Ihr Problembewusstsein zu schärfen.

\phantomsection\label{refs}
\begin{CSLReferences}{1}{0}
\bibitem[\citeproctext]{ref-wespe_mein_2022}
Wespe, L. (2022). \emph{Mein Leben und der Notenschnitt}. Voltafilm,
Alexa Meyer.
\url{https://www.voltafilm.ch/de/film/mein-leben-und-der-notenschnitt}

\end{CSLReferences}



\end{document}
